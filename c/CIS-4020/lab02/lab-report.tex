%%%%%%%%%%%%%%%%%%%%%%%%%%%%%%%%%%%%%%%%%%%%%%%%%%%%%%%%%%%%%%%%%%%%%%%%%%%%%%%%
% FILE    : lab-report.tex
% SUBJECT : CIS-4020 Lab 2 report
% AUTHOR  : Jeremy Audet <ichimonji10 at gmail dot com>
% LICENSE : Public Domain
%%%%%%%%%%%%%%%%%%%%%%%%%%%%%%%%%%%%%%%%%%%%%%%%%%%%%%%%%%%%%%%%%%%%%%%%%%%%%%%%

\documentclass{article}

\usepackage{fancyvrb}
\usepackage[pdftex]{graphicx}
\usepackage{listings}
\usepackage{hyperref}
\usepackage{url}

% The following are settings for the listings package.
% See the listings package documentation for more information.
\lstset{
    language=C,
    basicstyle=\small,
    stringstyle=\ttfamily,
    commentstyle=\ttfamily,
    xleftmargin=0.25in,
    showstringspaces=false
}

\begin{document}

\title{CIS-4020 Lab \#2\\TITLE}
\author{Jeremy Audet\thanks{Ichimonji10@gmail.com}\\
    Vermont Technical College}
    \date{September 05, 2013}
    \maketitle

\begin{abstract}
The purpose of this lab is to capture information from within the kernel
and write a module which exposes that information as a "proc" file.
\end{abstract}

\section{Introduction}
\label{sec:introduction}

The kernel makes several hundred system calls available to usermode programs.
These system calls provide programs the ability to perform actions such as
opening a file or determining their current process ID. These system calls are
typically exposed through standard C functions such as \texttt{getpid()}, but
they can be called more directly by including kernel headers and using the
\texttt{syscall()} function.

In this lab, the kernel is instructed to count how many times each system call
is used. A loadable module is then written which exposes this information to the
user as a proc file.

\section{Procedure}
\label{sec:procedure}

\begin{enumerate}
\item Edit \texttt{arch/x86/kernel/sys\_x86\_64.c}. Add the following below the
\texttt{\#include} statements:
\begin{lstlisting}
// see arch/x86/include/generated/asm/unistd_64.h
unsigned long long syscall_counts[313];
EXPORT_SYMBOL(syscall_counts);
\end{lstlisting}

\item Edit \texttt{arch/x86/kernel/entry\_64.S}. Add the following at line 531:
\begin{lstlisting}
// Increment syscall_counts[%rax]
incq syscall_counts(,%rax,8)
\end{lstlisting}

\item Compile the kernel, install it, and see if things are still working.  From
the root of the linux 3.6.11 source code:
\begin{Verbatim}
$ make
# make modules_install
# rsync -t arch/x86/boot/bzImage /boot/vmlinuz-3.6.11
# rsync -t .config /boot/config-3.6.11
# rsync -t System.map /boot/System.map-3.6.11
\end{Verbatim}

\item Flesh out the skeleton kernel module provided by Peter Chapin. The
interesting changes are shown in figures \ref{fig:start-function},
\ref{fig:next-function} and \ref{fig:show-function}.

\item Build the new module, insert it, and see what happens.
\begin{Verbatim}
$ ls
Kbuild  Makefile  syscall_counters.c
$ make
# insmod syscall_counters.ko
$ head /proc/syscall_counters
1 144825
2 55604
3 43613
4 48671
5 20792
6 27382
7 6413
8 67908
9 16230
10 38263
$ tail /proc/syscall_counters
304 0
305 0
306 0
307 0
308 0
309 0
310 0
311 0
312 0
313 0
# rmmod syscall_counters
\end{Verbatim}
\end{enumerate}

\section{Discussion}
\label{sec:discussion}

It is impossible to collect information about how many times each system call is
used with user-mode code. If the kernel developers had decided to provide a
callback mechanism which could call arbitrary code before each system call, this
might be possible, but no such mechanism has been made available. Instead, the
kernel itself must be modified to accomplish this task. This is why the
\texttt{syscall\_counts} array is declared and exported inside
\texttt{arch/x86/include/generated/asm/unisted\_64.h}.

With the \texttt{syscall\_counts} array in place, it must be populated. The
assembly code inserted into \texttt{arch/x86/kernel/entry\_64.s} accomplishes
this task.

The \texttt{counter\_seq\_start}\ref{fig:start-function},
\texttt{counter\_seq\_next}\ref{fig:next-function} and
\texttt{counter\_seq\_show}\ref{fig:show-function} functions extract information
from the \texttt{syscall\_counts} array and present it to the user. They use the
"seq api" to perform this task.

The first function returns a pointer to the Nth element of array
\texttt{syscall\_counts}. If the requested array element does not exist, it
returns \texttt{NULL}.

The second function behaves much like the first function, except that it also
increments the value pointed to by the \texttt{record\_number} variable.

The third function returns a formatted string containing a system call ID number
and the number of times that system call has been used. Finding the number of
times a given system call has been used is simple: just dereference the argument
\texttt{bookmark}, which is a pointer to a certain element of
\texttt{syscall\_counts}. Determining which system call ID is being interrogated
is trickier: the difference between \texttt{bookmark} and the beginning address
of \texttt{syscall\_counts} must be calculated, like this:
\begin{lstlisting}
int syscall_id = sc_location - syscall_counts + 1;
\end{lstlisting}

\section{Conclusion}
\label{sec:conclusion}

In this lab, I forced the kernel to track how many times each system call was
used. A loadable module was written which exposed this information to usermode
programs as a proc file.

\section{Appendix}
\label{sec:appendix}

\begin{figure}[tbhp] % place fig top, bottom, here, or alone on page
\label{fig:start-function}
\begin{lstlisting}[
    frame=single,
    caption={Start Function},
    xleftmargin=0in
]
// This function is called each time the application calls read(). It starts the
// process of accumulating data to fill the application buffer. Return a pointer
// representing the current item. Return NULL if there are no more items.
static void *counter_seq_start(struct seq_file *s, loff_t *record_number) {
    if(NUM_SYSCALLS <= *record_number) {
        return NULL;
    }
    return & syscall_counts[*record_number];
}
\end{lstlisting}
\end{figure}

\begin{figure}[tbhp] % place fig top, bottom, here, or alone on page
\label{fig:next-function}
\begin{lstlisting}[
    frame=single,
    caption={Next Function},
    xleftmargin=0in
]
// This function is called to compute the next record in the sequence given a
// pointer to the current record (in bookmark). It returns a pointer to the new
// record (essentially, an updated bookmark) and updates *record_number
// appropriately. Return NULL if there are no more items.
static void *counter_seq_next(
    struct seq_file *s,
    void *bookmark,
    loff_t *record_number
) {
    (*record_number) += 1;
    if(NUM_SYSCALLS <= *record_number) {
        return NULL;
    }
    return & syscall_counts[*record_number];
}
\end{lstlisting}
\end{figure}

\begin{figure}[tbhp] % place fig top, bottom, here, or alone on page
\label{fig:show-function}
\begin{lstlisting}[
    frame=single,
    caption={Show Function},
    xleftmargin=0in
]
// This function is called after next to actually compute the output. It can use
// various seq_... printing functions (such as seq_printf) to format the
// output. It returns 0 if successful or a negative value if it fails.
static int counter_seq_show(struct seq_file *s, void *bookmark) {
    // sc_location is the current location within `syscall_counts`
    unsigned long long * sc_location = (unsigned long long *)bookmark;
    int syscall_id = sc_location - syscall_counts + 1;
    if(0 > seq_printf(s, "%d %lld\n", syscall_id, *sc_location)) {
        return -1;
    }
    return 0;
}
\end{lstlisting}
\end{figure}

\end{document}
